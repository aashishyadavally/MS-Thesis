\chapter{}{Conclusion \& Future Directions}{Conclusion \& Future Directions}

\par The main purpose of this thesis was to develop machine learning models to effectively predict surface-level solar irradiance 24 hours into the future at multiple fixed and tracking solar arrays located at a solar farm near the University of Georgia. Towards this end, firstly, a study was conducted where the work done by Jones \cite{thesis_zach} was replicated. An input-selection scheme was designed to weed out the less relevant weather variables, and corresponding feature projections. This scheme helped in improving the performance (mean absolute error, $MAE$) in the replication study by 19.05\%, 19.68\% and 10.65\% (average across machine learning models) for the dual-axis tracking, fixed-axis and single-axis tracking solar arrays respectively. A best performance, i.e. least $MAE$ of 72.63 $W/m^2$, 44.94 $W/m^2$ and 63.60 $W/m^2$ was recorded for each of the arrays.

\par The effect of geographic expansion, i.e. including weather forecasts around the target location was evaluated. This was extended to $3$ x $3$ and $5$ x $5$ \textit{geo shapes}, by including the input-selected NAM weather forecasts from these additional cells. It was observed that an improvement in performance (marginal) with an increase in the \textit{geo shape} was seen for only the \textit{random forests} algorithm. By utilizing the weather forecast data corresponding to the $5$ x $5$ \textit{geo shape} as input for this machine learning technique, an $MAE$ of 69.38 $W/m^2$, 43.62 $W/m^2$ and 61.99 $W/m^2$ was recorded for the dual-axis tracking, fixed-axis and single-axis tracking solar arrays.

\par A few theory-driven bias-correction methodologies (multi-model blending approaches) were explored in Chapter 4. The motivation behind these approaches was to selectively correct the bias in global horizontal irradiance (GHI) reported in literature, by identifying the clear-sky conditions effectively. For this purpose, measures such as \textit{clear-sky index} and \textit{clearness index} were used depending on the empirical solar radiation model that the NAM Forecast System was being combined with. An exhaustive grid search was performed to identify a threshold for these measures, based on which the clear-sky conditions could be distinguished from cloudy-sky conditions easily. Based on these thresholds, the GHI weather variable was selectively corrected for the 18h NAM forecasts, for which an over-prediction of this parameter was observed. This was carried out by substituting the \restoregeometry\noindent GHI for such observations with the arithmetic mean of GHI from the NAM Forecast System and the corresponding empirical solar radiation model.

\par Such a bias-correction scheme resulted in an improvement in performance for the predictive models using the random forests technique, utilizing adjusted GHI from the model-blending techniques, with respect to just using the GHI from NAM Forecast System. A reduction in $MAE$ by 4.95\%, 4.53\% and 4.12\% for the dual-axis tracking, fixed axis and single-axis tracking solar arrays was observed for the model-blending methodology combining the NAM Forecast System and Clear-Sky Scaling technique, over using GHI from just the NAM Forecast System. Additionally, the model-blending methodology combining the NAM Forecast System with the Liu-Jordan model reduced the $MAE$ by 4.17\%, 4.14\% and 3.62\% for each of the solar arrays.

\par To further evaluate the model-blending approach, other NAM weather variables which were identified to be relevant, i.e. air temperature, height at planetary boundary layer and total cloud cover were included along with the adjusted GHI (obtained through the model-blending approaches). Select feature projections were chosen for each of the weather variables depending on the target hour offset in the forecast horizon, in line with the input-selection scheme described in 3.2. However, it was observed that such an input-selection scheme slightly depreciated the performance of the model-blending approaches combining the NAM Forecast System with both \textit{Clear-Sky Scaling} and \textit{Liu-Jordan} techniques.

\par The lack of improvement possibly indicates an inability to adequately identify sky conditions effectively, which in turn prevents accurate bias correction. In this work, the Ineichen model was used to determine the clear-sky GHI. It would be interesting to see if utilizing other empirical clear-sky models for this purpose will improve the ability of the \textit{clear-sky index} measure to identify the sky conditions, and in turn, improve model performance. This also leaves a scope for exploring superior techniques for distinguishing between sky conditions, and in turn, for identifying the over-prediction in the GHI weather variable.

\par In the theory-based bias correction methodology described in this work, a simple bias-correction function was used, wherein, the GHI from the NAM Forecast System was substituted with the arithmetic mean of GHI from the NAM Forecast System and the corresponding empirical solar radiation model. This can be improved upon by subjecting both of these GHI estimates to statistical post-processing, and determining a superior bias-correction function. This can be investigated in future work.

\par By utilizing the meteorological projections in NAM Forecast System, \textit{clear-sky index} was projected into the future as well. Predictive models were developed utilizing this measure rather than GHI, in order to exploit it's presumed ability to capture seasonality. A best performance, i.e. least $MAE$ of 79.58 $W/m^2$, 49.18 $W/m^2$ and 69.98 $W/m^2$ was recorded for the dual-axis tracking, fixed-axis and single-axis tracking solar arrays respectively. In order to assess the seasonality-capturing ability, a stratified seasonal analysis was performed, where the performance of individual forecasts across seasons (summer, spring, winter, autumn) was compared with that of the models utilizing GHI. It was observed that the former performed poorly across seasons, when compared to that of the latter. Consequently, it can be concluded that the presumed cyclicity-capturing ability of \textit{clear-sky index} did not translate into improving the performance of the predictive models. It would be interesting to explore other clear-sky models in literature towards determining this measure, and reviewing their performance in such a framework. This can be looked into in further work.

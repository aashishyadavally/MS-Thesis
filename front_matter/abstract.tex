% Hide page numbers for this page.
\thispagestyle{empty}
\newgeometry{top=1.75in, left=1.5in, right=1in, bottom=1in}
\begin{center}
    EXPLORING MACHINE LEARNING BASED DAY-AHEAD SOLAR IRRADIANCE FORECASTING METHODOLOGIES
    
    by
    
    AASHISH YADAVALLY
    
    (Under the Direction of Frederick Maier)
    
    ABSTRACT
\end{center}

Predicting solar irradiance is an important topic in renewable energy generation, and leveraging weather forecast data released by meteorological organizations towards building machine learning models for this purpose is a common strategy. In this work, the North America Mesoscale (NAM) Forecast Model data was combined with ground-based solar irradiance data at the solar farm in the University of Georgia, towards predicting day-ahead solar irradiance at a one-hour temporal resolution. The efficacy of a proposed input-selection scheme was tested, and it was observed that the scheme helped decreasing the mean absolute error (MAE) across models by 19.05\%, 19.68\%, 10.65\% for the dual-axis tracking, fixed-axis and single-axis tracking solar arrays respectively. The affect of spatial expansion on the attribute-selected weather forecast data was investigated, and it was observed that such an expansion only marginally improved, or resulted in a degradation in performance of the models. Furthermore, a multi-model blending approach was investigated wherein the NAM Forecast Model was combined with empirical solar radiation models such as \textit{Clear-Sky Scaling} and \textit{Liu-Jordan Model}. Such an approach only marginally improved the performance of the models. In addition, the efficacy of predictive models involving clear-sky index, which is known to capture seasonal patterns was evaluated.

\vspace{\baselineskip}
\noindent
INDEX WORDS: solar irradiance forecasting, machine learning, numerical weather prediction, clear-sky scaling, liujordan, clear-sky index, clearness index

\newpage
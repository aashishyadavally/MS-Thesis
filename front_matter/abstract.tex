% Hide page numbers for this page.
\thispagestyle{empty}
\newgeometry{top=1.75in, left=1.5in, right=1in, bottom=1in}
\begin{center}
    AN EXPLORATION OF MACHINE LEARNING BASED DAY-AHEAD SOLAR IRRADIANCE FORECASTING METHODOLOGIES
    
    by
    
    AASHISH YADAVALLY
    
    (Under the Direction of Frederick Maier)
    
    ABSTRACT
\end{center}
Predicting solar irradiance is an important topic in renewable energy generation. In this work, the North American Mesoscale (NAM) Forecast System data is augmented with irradiance observations from the solar farm at the University of Georgia, towards forecasting 24 hours into the future. For the machine learning models used for this purpose, an input-selection scheme is presented and evaluated. This scheme significantly improved the performance, and resulted in a mean absolute error ($MAE$) of 72.63$W/m^2$, 44.94$W/m^2$ and 63.60$W/m^2$ for the dual-axis tracking, fixed-axis and single-axis tracking solar arrays respectively. The effect of geographic expansion, by including additional weather forecasts is evaluated. Furthermore, to correct the reported bias in global horizontal irradiance (GHI) in NAM Forecast System, theory-driven bias-correction approaches are explored, where NAM Forecast System is selectively combined with \textit{Clear-Sky Scaling} and \textit{Liu-Jordan} techniques. In addition, the ability of predictive models involving clear-sky index to capture seasonal patterns is evaluated.

\vspace{3\baselineskip}
\noindent
INDEX WORDS: solar forecasting, machine learning, numerical weather prediction, bias correction, clear-sky index, clearness index

\newpage
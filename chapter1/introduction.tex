\chapter{}{Introduction}{Introduction}
\par Fossil fuels are the dominant cause of climate change, and transitioning from energy sources depending on them to renewable forms is one of the most powerful ways in which we can reduce our ecological footprint as a society. However, owing to the unpredictability associated with carbon-free sources such as solar energy and wind energy, incorporating them into an electrical energy system is challenging. Thus, to ensure a balance between the consumption and production of solar energy, accurate prediction of solar irradiance is of utmost importance.

\par Solar irradiance forecasting can be performed by several methods depending on the temporal variability of the forecast horizon, ranging from a minute to several days \cite{multimodel_figure1}. Generally, the predictions worsen as the forecast horizon increases. Numerical weather prediction (NWP) models utilize mathematical models of atmosphere and ocean systems to predict weather variables from hours to months in advance. Thus, using them for day-ahead solar forecasting is a common strategy. The National Oceanic and Atmospheric Administration (NOAA) maintains various global and mesoscale weather prediction models for weather forecasting. Mesoscale models are three-dimensional regional models based on thermodynamic equations describing physical processes, which incorporate the inherent unpredictability of many small-scale phenomena. The North American Mesoscale (NAM) Forecast System \cite{litrev_nwp_nam} is one of the major mesoscale-based weather forecast models maintained by NOAA.

\par The specific purpose of this work is to develop machine learning models to effectively predict surface-level solar irradiance 24 hours into the future at multiple fixed and tracking solar arrays located at a solar farm near the University of Georgia. The developed models rely heavily on the use of NAM weather forecasts, which are released four times a day at six-hour intervals (00h, 06h, 12h, 18h UTC) for a grid of $12km$ x $12km$ cells covering the continental United States. NAM forecasts from the years 2017 and 2018 were used to develop and evaluate these models. 

\restoregeometry
\par The current work extends prior work undertaken at the University of Georgia related to solar irradiance forecasting (which have been discussed in Chapter 2). Of special importance is the work of Jones \cite{thesis_zach}, who developed a machine learning pipeline to forecast day-ahead solar irradiance using the weather forecast data from the NAM Forecast System. Chapter 3 begins by replicating these results. The models developed by Jones \cite{thesis_zach} utilized nine weather variables from the NAM data. Importantly, however, \textit{total cloud cover}, which is defined as the fraction of the sky covered by visible clouds was not considered. Cloudiness is considered to be an important meteorological factor in determining the amount of solar radiation reaching earth's surface. Thus, all the ten weather variables were further analyzed with respect to the target irradiance observations recorded at the solar farm, intending to quantify their usefulness in the day-ahead irradiance prediction. It was found that surface temperature, global horizontal irradiance, total cloud cover and atmospheric height were more significant than others.

\par The NAM Forecast System predicts values for weather variables 84 hours into the future. These are referred to as feature projections. The first 37 feature projections are at a one-hour temporal resolution (starting at the zero-hour or reference time). Feature projections corresponding to the subsequent 48 hours are reported at a three-hour temporal resolution. In his work, Jones \cite{thesis_zach} used all 37 feature projections at a one-hour temporal resolution, for the nine weather variables as predictors to the machine learning models.

\par In the current work, it was hypothesized that this was unnecessary and needlessly increases model training time. To investigate this, the relationship between the first 25 feature projections of GHI in NAM forecasts (starting at the zero-hour or reference time), was studied with respect to the irradiance observations from the fixed-axis solar array in the forecast horizon. Using a mutual information matrix, we concluded that the target irradiance observations for a particular target hour did not depend on all the feature projections of the NAM weather variable. Because of this, only data from 6 hours prior to the target hour to 6 hours ahead of the target hour were chosen as predictors in the models developed. This input-selection scheme helped in reducing the computational cost of training the machine learning models significantly. 

\par This input-selection scheme was used in a series of experiments to develop models analogous to those of Jones \cite{thesis_zach}. With respect to the performance of the latter, an average improvement in performance (across different machine learning models) by 19.05\%, 19.68\% and 10.65\% was recorded for the dual-axis tracking, fixed-axis and single-axis tracking solar arrays. We attribute this improvement in performance to the weeding out of less relevant features with the help of the input-selection scheme. The best performance was recorded using the \textit{random forest} machine learning technique, with a mean absolute error (MAE) of 72.63 $W/m^2$, 44.94 $W/m^2$ and 63.60 $W/m^2$ respectively for each of the solar arrays. 

\par Another series of experiments was performed to test the effect of geographic expansion, i.e. including a larger geographic area from the NAM forecast as input. The NAM Forecast System generates multiple grids of weather forecasts, where each cell corresponds to a $12km$ x $12km$ geographical area. In \cite{thesis_zach}, Jones included NAM forecasts corresponding to a grid of cells surrounding the cell representing Athens. He noted that considering a $3$ x $3$ grid of cells was optimal, as the improvement in performance diminished for even greater grid sizes. We also investigated the significance of similar geographic expansion, but on the NAM data obtained by incorporating the input-selection scheme. It is observed that such an expansion does not significantly improve the solar irradiance predictions. In fact, in a few cases, the geographic expansion had a detrimental effect on the performance of the machine learning models.

\par Beyond their overall performance, the performance of the machine learning models was investigated in a stratified manner. The predictions of each of the 00h, 06h, 12h and 18h NAM weather forecasts were analyzed independently. For each, a \textit{diurnal} analysis was performed, wherein the performance of the models for each target hour in the forecast horizon was compared. Such an analysis helped in understanding the ability of the machine learning models to gauge the diurnal characteristics.

\par Among the variables modelled by the NAM Forecast System is \textit{downward short-wave radiation flux} (also called global horizontal irradiance, GHI \cite{multimodel_overpredict}). It is an estimate of the total amount of short-wave radiation that reaches the Earth's surface, and is essential for short-term solar forecasting. It has been reported in literature that the NAM Forecast System tends to overpredict GHI when visible clouds are not present in the sky \cite{multimodel_overpredict}. In order to correct such a bias, identifying the amount of clouds in the sky becomes essential. This can be estimated using empirical solar radiation models, which formulate relations between different meteorological variables through experimental observations. In Chapter 4, theory-driven bias correction methodologies were explored, which involved blending the physical NAM Forecast System with such solar radiation models, so as to selectively correct the bias in GHI.

\vspace{0.5cm}
\par From among the different empirical formulations proposed to estimate GHI from environmental conditions, \textit{Clear-sky Scaling} \cite{pvlib_ineichen} and \textit{Liu-Jordan} \cite{pvlib_liujordan2} techniques were studied. For the purpose of correcting the bias in the GHI estimates from the NAM Forecast System, the GHI retrieved through each of these empirical solar radiation models was combined with the GHI from the NAM Forecast System depending on metrics such as \textit{clear-sky index} and \textit{clearness index}, which are different measures for estimating the amount of cloudiness in the sky.

\par A series of experiments was conducted using the \textit{random forest} algorithm to test the effectiveness of such a model-blending approach. Three variants of NAM data was input to these machine learning models: GHI from the NAM Forecast System ($GHI_\textsubscript{NAM}$); adjusted GHI, obtained from blending NAM Forecast System with \textit{Clear-Sky Scaling} technique (GHI\textsubscript{NAM+CS}); adjusted GHI, obtained by blending NAM Forecast System with \textit{Liu-Jordan} model (GHI\textsubscript{NAM+LJ}). The performance of the \textit{random forests} utilizing each of the adjusted GHI variants was compared with \textit{random forests} utilizing $GHI_\textsubscript{NAM}$. It was observed that the blending methodology involving NAM Forecast System and \textit{Clear-Sky Scaling} resulted in an improvement in performance (decrease in $MAE$) by 4.95\%, 4.53\% and 4.12\% for the dual-axis tracking, fixed-axis and single-axis tracking solar arrays respectively. In comparison, the blending methodology involving NAM Forecast System and \textit{Liu-Jordan} model recorded an improvement in performance by 4.17\%, 4.14\% and 3.62\% for each of the solar arrays.

\par The above experiments were only based on GHI, however. That is, GHI (corrected or uncorrected) was the only predictor used to develop these models. A new series of experiments was conducted by including the other weather variables from NAM Forecast System such as \textit{air temperature}, \textit{total cloud cover} and \textit{atmospheric height} along with the three variants of GHI used in the previous set of experiments. The input-selection scheme described earlier was incorporated into this weather forecast data, and select feature projections depending on the target hour in forecast horizon was selected for each of these weather variables. In this case however, the model-blending methodology involving NAM Forecast System performed slightly better than both of the blending methodologies. For the model-blending methodology involving \textit{Clear-Sky Scaling} technique, an $MAE$ of 72.57 $W/m^2$, 44.91 $W/m^2$ and 63.56 $W/m^2$ was recorded for the dual-axis tracking, fixed-axis and single-axis tracking solar arrays respectively. For the blending methodology involving \textit{Liu-Jordan} model, an $MAE$ of 72.74 $W/m^2$, 45.25 $W/m^2$ and 63.97 $W/m^2$ was recorded for each of the solar arrays.
\vspace{1.25em}

\par\textit{Clear-sky index} (which is computed using various meteorological variables such as GHI, solar position and solar zenith angle) is known to capture the diurnal and seasonal trends in weather data effectively. Because of this, it was suspected that the clear-sky index might improve model performance in comparison to using GHI. Thus, another series of experiments was conducted by developing predictive models utilizing \textit{clear-sky index} (in place of GHI), along with air temperature, total cloud cover and atmospheric height. Select feature projections of each of these weather variables were used as predictors to the machine learning models.

\par However, it was observed that the performance of these predictive models paled in comparison to those utilizing the input-selected weather forecast data including $GHI\textsubscript{NAM}$. An $MAE$ of 79.58 $W/m^2$, 49.18 $W/m^2$ and 69.98 $W/m^2$ was recorded for the dual-axis tracking, fixed-axis and single-axis tracking solar arrays respectively. An analysis of the performance of individual 12h and 18h NAM forecasts was conducted. The $MAE$ corresponding to the \textit{spring} and \textit{summer} season increased for the both the forecasts, with respect to the predictive models utilizing $GHI\textsubscript{NAM}$. Consequently, it can be concluded that the presumed ability of \textit{clear-sky index} to capture the diurnal and seasonal trends did not translate into improving the performance of the predictive models.

\par The theory-driven bias correction methodology undertaken in this work solely corrects the bias in GHI, and doesn't address the bias correction in other weather variables. In addition, the lack of improvement in the performance of predictive models upon including additional weather variables possibly indicates the inability of the models to identify the clear-sky conditions effectively. This, in turn prevents accurate bias correction in GHI. Future work can explore superior approaches for detecting clear-sky conditions, which will improve upon the bias correction in GHI, as well as other weather variables. Furthermore, it was observed that the predictive models utilizing clear-sky index performed worse than those utilizing GHI across all seasons in the year. We note that there are other clear-sky models in literature which can conceivably improve the ability of clear-sky index to capture such seasonal trends. These models can be investigated in further work.

\newpage
\chapter{}{Introduction}{Introduction}
\par Climate change is widely considered to be a major environmental crisis that we face today. Transitioning from fossil fuel based energy sources to renewable energy sources is one of the most powerful ways in which we can reduce our ecological footprint as a society. However, owing to the unpredictability associated with carbon-free sources such as solar energy and wind energy, incorporating them into an electrical energy structure is challenging. The unpredictable nature of solar radiation also leads to a number of other problems such as voltage fluctuations, stability issues, etc. Thus, to ensure a balance between the consumption and production of solar energy, accurate prediction of solar irradiance is of utmost importance, as this is a measure of the available fuel of a solar power generator at a given future moment in time. \cite{exec_summary_1}

\par Solar irradiance forecasting can be performed by several methods depending on the temporal variability of the forecast horizon, ranging from a minute to several days. However, the predictions worsen as the forecast horizon increases. Numerical weather prediction (NWP) models use mathematical models of atmosphere and oceans to predict weather variables up to two days and beyond. Thus, using such physical weather forecast models for day-ahead solar forecasting is a common strategy. National Oceanic and Atmospheric Administration (NOAA) maintains various global and mesoscale weather prediction models for weather forecasting. Mesoscale models are three-dimensional regional models based on thermodynamic equations describing the physical processes. They are efficient in predicting smaller scale detail. North America Mesoscale (NAM) Forecast System \cite{litrev_nwp_nam} is one of the major mesoscale-based weather forecast models maintained by NOAA.

\par The purpose of this work is to effectively predict surface-level solar irradiance 24 hours into the future, at different fixed and tracking solar arrays located at a solar farm near the University of Georgia. Towards accomplishing this, NAM weather forecasts were used for solar irradiance forecasting, which is an integral part of this work. The NAM Forecast System releases four forecasts in a day at six-hour intervals (00h, 06h, 12h, 18h UTC) for multiple $12km$ x $12km$ grids spanning \restoregeometry\noindent the continental United States. NAM data for the years 2017 and 2018 were collected and analyzed.

\par Among the different weather variables modeled by NAM Forecast System, downward short-wave radiation flux parameter (synonymous with global horizontal irradiance, GHI \cite{multimodel_overpredict}), which is an estimate of the total amount of short-wave radiation (radiation with wavelength between $0.2\mu m$ and $3.0 \mu m$) that reaches the Earth's surface, is essential for short-term solar forecasting. Thus, GHI was post-processed with respect to the surface-level solar irradiance using multiple machine learning models. Furthermore, site-specific information was incorporated into the models by including additional weather variables such as surface temperature, total cloud cover and atmospheric height. However, one major drawback in this setup was that the GHI measured by the NAM Forecast System is  variable, depending on the cloud conditions. Thus, the need to empirically correct the bias in this parameter was identified and techniques in this regard were explored.

\par The radiation from the sun reaching the earth surface is a combination of direct normal irradiance (DNI) and diffused horizontal irradiance (DHI), which together amount to the global horizontal irradiance (GHI). Of the global solar radiation reaching the earth's surface, DNI is the direct radiance received on a plane perpendicular to the earth's surface. DHI is the fraction of global solar radiation scattered by particles in the atmosphere. GHI, in turn, is the fraction of global solar radiation reaching the earth's surface on a plane horizontal to the surface. Several formulations have been proposed to estimate these metrics from environmental conditions. Of these, the \textit{Clear-sky Scaling} technique \cite{pvlib_ineichen} and \textit{Liu-Jordan} \cite{pvlib_liujordan2} method were studied. The irradiance metrics retrieved through these formulations were used to correct the bias in the GHI estimate from NAM Forecast System. 

\par This thesis documents various experiments we conducted towards improving the efficacy of solar irradiance prediction, and can be divided into four chapters (1 through 4) in this fashion: Chapter 1 discusses the problem formulation and motivation for the approach we undertook to forecast solar irradiance. In Chapter 2, several research works were reviewed which discuss the methodologies undertaken by various solar forecasting researchers depending on the forecast horizon of the predictions or the spatial resolution of the input data. Major empirical solar radiation models which formulate the irradiance metrics described above, and which can be broadly classified into \textit{decomposition models} and \textit{parametric models} are briefly discussed. Furthermore, recent works undertaken at the University of Georgia, related to solar irradiance forecasting are reviewed, as this lays a basis to build upon and compare the results obtained in this work.

\par Jones et al. \cite{thesis_zach} developed a machine learning pipeline to forecast day-ahead solar irradiance using the weather forecast data from the NAM Forecast System. Chapter 2 begins by replicating the results obtained in \cite{thesis_zach}, which used nine weather variables from the NAM data. However, they did not include the \textit{total cloud cover} parameter, which is defined as the fraction of the sky covered by visible clouds. Cloudiness is considered to be one of the most important meteorological factors determining the amount of solar radiation reaching the earth surface. Thus, all the ten features were further analyzed with respect to the target irradiance observations recorded at the solar farm, intending to quantify their usefulness in the day-ahead irradiance prediction. It was identified that weather variables such as surface temperature, global horizontal irradiance, total cloud cover and atmospheric height were more significant. 

\par Each of the weather variables in the NAM data has 36 feature projections at a 1-hour temporal resolution, and the remaining at a 3-hour temporal resolution. Jones et al. \cite{thesis_zach} used all 36 feature projections at 1-hour temporal resolution for the nine weather variables as predictors for machine learning models. To rationalize this scheme, 24 feature projections of the GHI weather variable were compared with irradiance observations from the fixed solar array for all 24 target hours in the forecast horizon. It was observed that the target irradiance observations for a particular target hour did not depend on all the feature projections of the weather variable. Thus, 13 feature projections (from 6 hours prior to the target hour to 6 hours ahead of the target hour) of all weather variables were chosen as predictors for each of the target hours in the forecast horizon separately. This input-selection scheme helped in reducing the computational cost of training the machine learning models significantly. In one series of experiments, results obtained through this scheme were compared with those obtained by replicating the methodology followed in \cite{thesis_zach}. An average improvement in performance (across different machine learning models) by \_\_\_ was recorded, with the best machine learning model having a mean absolute error (MAE) of $\_\_ W/m^2$, $\_\_ W/m^2$, $\_\_ W/m^2$ for the dual-axis tracking solar array, fixed-axis solar array and single-axis tracking solar array respectively.

\par Another series of experiments were performed to test the effect of the geographic expansion of forecast coverage by including additional weather forecasts from area surrounding the target location. In \cite{thesis_zach}, Jones et al. included weather forecasts from a grid of cells surrounding the NAM data grid representing Athens. They noted that considering a $3 x 3$ grid of cells was most appropriate towards the solar irradiance prediction, and that the geographic expansion had diminishing returns for grid sizes greater than $3 x 3$. However in this work, it was observed that such a geographic expansion following the input selection scheme described earlier did not help in improving the solar irradiance predictions, and the $1 x 1$ grid, i.e. just considering the NAM data grid representing Athens was more appropriate. In fact, in a few cases, the geographic expansion also had a detrimental effect in performance.

\par Furthermore, the analysis of the machine learning models was extended in two ways: a stratified analysis for each of the forecast hours was performed, wherein the ability of the models to predict the diurnal aspects of the forecasts was gauged; a seasonal analysis was performed, wherein the ability of the models to realize the seasonal characteristics was gauged. Considering the target location is in Georgia, United States, which experiences four major seasons i.e. \textit{spring}, \textit{summer}, \textit{autumn} and \textit{winter}, the seasonal analysis was performed for each of these seasons exclusively. 

\par In Chapter 4, a multi-model blending approach is introduced to address the bias in GHI recorded by NAM Forecast System. In this approach, the GHI recorded by NAM Forecast System is combined with GHI formulated with the help of empirical solar radiation models such as \textit{Clear-sky Scaling} and \textit{Liu-Jordan Model} based on metrics such as clear-sky index ($k_c$) and clearness index ($k_t$). The clear-sky index can be defined as the ratio of measured GHI ($GHI_{NAM}$) to GHI in clear-sky conditions ($GHI_{CS}$), estimated through the \textit{clear-sky scaling} technique. This measure was estimated by truncating the ratio to values between 0 and 2, 0 indicating overcast conditions and 2 indicating a clear day, i.e. sky conditions defined by the absence of visible clouds. It was observed that there was a high variability in this parameter for only the 18h NAM forecasts from among the 00h, 06h, 12h and 18h forecasts released by the NAM Forecast System. Thus for all 18h forecasts with $k_c > 1.5$, $GHI_{NAM}$ and $GHI_{CS}$ were averaged. This resulted in an average improvement in performance (across different machine learning models) by \_\_\_\_\_, the best performing model having a mean absolute error (MAE) of $\_\_ W/m^2$, $\_\_ W/m^2$, $\_\_ W/m^2$ for the dual-axis tracking solar array, fixed-axis solar array and single-axis tracking solar array respectively.

\par A similar model blending approach was carried out between the \textit{Liu-Jordan} model and the NAM Forecast System based on the \textit{clearness index} metric. Clearness index is a measure of the clearness in atmosphere, and can be defined as the ratio between the solar radiation transmitted through the atmosphere to the solar radiation reaching the earth's surface. The clearness index measure is critical in the formulation of irradiance metrics using \textit{Liu-Jordan} model, and thus, using it to contrast the GHI recorded by NAM Forecast System and Liu-Jordan model is intuitive. Thus for all 12h and 18h forecasts, for which GHI is significant, $k_t$ was estimated. In order to gauge the variability of the clearness index better, it was truncated to values between 0 (overcast conditions) and 2 (clear atmosphere). For the forecasts with $k_t > 0.7$, $GHI_{NAM}$ and $GHI_{LJ}$ were averaged. DHI estimated by the \textit{Liu-Jordan} model was identified to be an important parameter in solar irradiance predictions as well. Thus, it was included along with the weather parameters from the NAM Forecast System as predictors for the different machine learning models. This methodology resulted in an average improvement in performance (across different machine learning models) by \_\_\_\_\_, the best performing model having a mean absolute error (MAE) of $\_\_ W/m^2$, $\_\_ W/m^2$, $\_\_ W/m^2$ for the dual-axis tracking solar array, fixed-axis solar array and single-axis tracking solar array respectively.

\newpage